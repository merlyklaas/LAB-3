% Options for packages loaded elsewhere
\PassOptionsToPackage{unicode}{hyperref}
\PassOptionsToPackage{hyphens}{url}
%
\documentclass[
]{article}
\usepackage{lmodern}
\usepackage{amssymb,amsmath}
\usepackage{ifxetex,ifluatex}
\ifnum 0\ifxetex 1\fi\ifluatex 1\fi=0 % if pdftex
  \usepackage[T1]{fontenc}
  \usepackage[utf8]{inputenc}
  \usepackage{textcomp} % provide euro and other symbols
\else % if luatex or xetex
  \usepackage{unicode-math}
  \defaultfontfeatures{Scale=MatchLowercase}
  \defaultfontfeatures[\rmfamily]{Ligatures=TeX,Scale=1}
\fi
% Use upquote if available, for straight quotes in verbatim environments
\IfFileExists{upquote.sty}{\usepackage{upquote}}{}
\IfFileExists{microtype.sty}{% use microtype if available
  \usepackage[]{microtype}
  \UseMicrotypeSet[protrusion]{basicmath} % disable protrusion for tt fonts
}{}
\makeatletter
\@ifundefined{KOMAClassName}{% if non-KOMA class
  \IfFileExists{parskip.sty}{%
    \usepackage{parskip}
  }{% else
    \setlength{\parindent}{0pt}
    \setlength{\parskip}{6pt plus 2pt minus 1pt}}
}{% if KOMA class
  \KOMAoptions{parskip=half}}
\makeatother
\usepackage{xcolor}
\IfFileExists{xurl.sty}{\usepackage{xurl}}{} % add URL line breaks if available
\IfFileExists{bookmark.sty}{\usepackage{bookmark}}{\usepackage{hyperref}}
\hypersetup{
  pdftitle={LAB 3},
  pdfauthor={Merly Klaas},
  hidelinks,
  pdfcreator={LaTeX via pandoc}}
\urlstyle{same} % disable monospaced font for URLs
\usepackage[margin=1in]{geometry}
\usepackage{color}
\usepackage{fancyvrb}
\newcommand{\VerbBar}{|}
\newcommand{\VERB}{\Verb[commandchars=\\\{\}]}
\DefineVerbatimEnvironment{Highlighting}{Verbatim}{commandchars=\\\{\}}
% Add ',fontsize=\small' for more characters per line
\usepackage{framed}
\definecolor{shadecolor}{RGB}{248,248,248}
\newenvironment{Shaded}{\begin{snugshade}}{\end{snugshade}}
\newcommand{\AlertTok}[1]{\textcolor[rgb]{0.94,0.16,0.16}{#1}}
\newcommand{\AnnotationTok}[1]{\textcolor[rgb]{0.56,0.35,0.01}{\textbf{\textit{#1}}}}
\newcommand{\AttributeTok}[1]{\textcolor[rgb]{0.77,0.63,0.00}{#1}}
\newcommand{\BaseNTok}[1]{\textcolor[rgb]{0.00,0.00,0.81}{#1}}
\newcommand{\BuiltInTok}[1]{#1}
\newcommand{\CharTok}[1]{\textcolor[rgb]{0.31,0.60,0.02}{#1}}
\newcommand{\CommentTok}[1]{\textcolor[rgb]{0.56,0.35,0.01}{\textit{#1}}}
\newcommand{\CommentVarTok}[1]{\textcolor[rgb]{0.56,0.35,0.01}{\textbf{\textit{#1}}}}
\newcommand{\ConstantTok}[1]{\textcolor[rgb]{0.00,0.00,0.00}{#1}}
\newcommand{\ControlFlowTok}[1]{\textcolor[rgb]{0.13,0.29,0.53}{\textbf{#1}}}
\newcommand{\DataTypeTok}[1]{\textcolor[rgb]{0.13,0.29,0.53}{#1}}
\newcommand{\DecValTok}[1]{\textcolor[rgb]{0.00,0.00,0.81}{#1}}
\newcommand{\DocumentationTok}[1]{\textcolor[rgb]{0.56,0.35,0.01}{\textbf{\textit{#1}}}}
\newcommand{\ErrorTok}[1]{\textcolor[rgb]{0.64,0.00,0.00}{\textbf{#1}}}
\newcommand{\ExtensionTok}[1]{#1}
\newcommand{\FloatTok}[1]{\textcolor[rgb]{0.00,0.00,0.81}{#1}}
\newcommand{\FunctionTok}[1]{\textcolor[rgb]{0.00,0.00,0.00}{#1}}
\newcommand{\ImportTok}[1]{#1}
\newcommand{\InformationTok}[1]{\textcolor[rgb]{0.56,0.35,0.01}{\textbf{\textit{#1}}}}
\newcommand{\KeywordTok}[1]{\textcolor[rgb]{0.13,0.29,0.53}{\textbf{#1}}}
\newcommand{\NormalTok}[1]{#1}
\newcommand{\OperatorTok}[1]{\textcolor[rgb]{0.81,0.36,0.00}{\textbf{#1}}}
\newcommand{\OtherTok}[1]{\textcolor[rgb]{0.56,0.35,0.01}{#1}}
\newcommand{\PreprocessorTok}[1]{\textcolor[rgb]{0.56,0.35,0.01}{\textit{#1}}}
\newcommand{\RegionMarkerTok}[1]{#1}
\newcommand{\SpecialCharTok}[1]{\textcolor[rgb]{0.00,0.00,0.00}{#1}}
\newcommand{\SpecialStringTok}[1]{\textcolor[rgb]{0.31,0.60,0.02}{#1}}
\newcommand{\StringTok}[1]{\textcolor[rgb]{0.31,0.60,0.02}{#1}}
\newcommand{\VariableTok}[1]{\textcolor[rgb]{0.00,0.00,0.00}{#1}}
\newcommand{\VerbatimStringTok}[1]{\textcolor[rgb]{0.31,0.60,0.02}{#1}}
\newcommand{\WarningTok}[1]{\textcolor[rgb]{0.56,0.35,0.01}{\textbf{\textit{#1}}}}
\usepackage{graphicx,grffile}
\makeatletter
\def\maxwidth{\ifdim\Gin@nat@width>\linewidth\linewidth\else\Gin@nat@width\fi}
\def\maxheight{\ifdim\Gin@nat@height>\textheight\textheight\else\Gin@nat@height\fi}
\makeatother
% Scale images if necessary, so that they will not overflow the page
% margins by default, and it is still possible to overwrite the defaults
% using explicit options in \includegraphics[width, height, ...]{}
\setkeys{Gin}{width=\maxwidth,height=\maxheight,keepaspectratio}
% Set default figure placement to htbp
\makeatletter
\def\fps@figure{htbp}
\makeatother
\setlength{\emergencystretch}{3em} % prevent overfull lines
\providecommand{\tightlist}{%
  \setlength{\itemsep}{0pt}\setlength{\parskip}{0pt}}
\setcounter{secnumdepth}{-\maxdimen} % remove section numbering

\title{LAB 3}
\author{Merly Klaas}
\date{}

\begin{document}
\maketitle

\hypertarget{r-markdown}{%
\subsection{R Markdown}\label{r-markdown}}

This is an R Markdown document. Markdown is a simple formatting syntax
for authoring HTML, PDF, and MS Word documents. For more details on
using R Markdown see \url{http://rmarkdown.rstudio.com}.

\hypertarget{part-a-foundations}{%
\section{Part A: Foundations}\label{part-a-foundations}}

\hypertarget{briefly-name-and-describe-the-three-fundamental-components-of-a-function.}{%
\subsection{Briefly name and describe the three fundamental components
of a
function.}\label{briefly-name-and-describe-the-three-fundamental-components-of-a-function.}}

\begin{itemize}
\item
  \begin{enumerate}
  \def\labelenumi{\arabic{enumi}.}
  \tightlist
  \item
    The formals(), the list of arguments that control how you call the
    function.
  \end{enumerate}
\item
  \begin{enumerate}
  \def\labelenumi{\arabic{enumi}.}
  \setcounter{enumi}{1}
  \tightlist
  \item
    The body(), the code inside the function.
  \end{enumerate}
\item
  \begin{enumerate}
  \def\labelenumi{\arabic{enumi}.}
  \setcounter{enumi}{2}
  \tightlist
  \item
    The environment(), the data structure that determines how the
    function finds the values associate with the names.
  \end{enumerate}
\end{itemize}

\hypertarget{describe-three-different-ways-functions-can-be-storedapplied-and-an-example-of-when-you-might-want-to-use-each-version.}{%
\subsection{Describe three different ways functions can be
stored/applied and an example of when you might want to use each
version.}\label{describe-three-different-ways-functions-can-be-storedapplied-and-an-example-of-when-you-might-want-to-use-each-version.}}

\hypertarget{create-a-function-object-with-function-and-bind-it-to-a-name-with--}{%
\section{1. Create a function object (with function) and bind it to a
name with
\textless-}\label{create-a-function-object-with-function-and-bind-it-to-a-name-with--}}

\begin{Shaded}
\begin{Highlighting}[]
\NormalTok{f01 <-}\StringTok{ }\ControlFlowTok{function}\NormalTok{(x) \{}
  \KeywordTok{sin}\NormalTok{(}\DecValTok{1} \OperatorTok{/}\StringTok{ }\NormalTok{x }\OperatorTok{^}\StringTok{ }\DecValTok{2}\NormalTok{)}
\NormalTok{\}}
\end{Highlighting}
\end{Shaded}

\hypertarget{anonymous-function.}{%
\section{2. Anonymous function.}\label{anonymous-function.}}

\begin{Shaded}
\begin{Highlighting}[]
\KeywordTok{lapply}\NormalTok{(mtcars, }\ControlFlowTok{function}\NormalTok{(x) }\KeywordTok{length}\NormalTok{(}\KeywordTok{unique}\NormalTok{(x)))}
\end{Highlighting}
\end{Shaded}

\begin{verbatim}
## $mpg
## [1] 25
## 
## $cyl
## [1] 3
## 
## $disp
## [1] 27
## 
## $hp
## [1] 22
## 
## $drat
## [1] 22
## 
## $wt
## [1] 29
## 
## $qsec
## [1] 30
## 
## $vs
## [1] 2
## 
## $am
## [1] 2
## 
## $gear
## [1] 3
## 
## $carb
## [1] 6
\end{verbatim}

\begin{Shaded}
\begin{Highlighting}[]
\KeywordTok{Filter}\NormalTok{(}\ControlFlowTok{function}\NormalTok{(x) }\OperatorTok{!}\KeywordTok{is.numeric}\NormalTok{(x), mtcars)}
\end{Highlighting}
\end{Shaded}

\begin{verbatim}
## data frame with 0 columns and 32 rows
\end{verbatim}

\begin{Shaded}
\begin{Highlighting}[]
\KeywordTok{integrate}\NormalTok{(}\ControlFlowTok{function}\NormalTok{(x) }\KeywordTok{sin}\NormalTok{(x) }\OperatorTok{^}\StringTok{ }\DecValTok{2}\NormalTok{, }\DecValTok{0}\NormalTok{, pi)}
\end{Highlighting}
\end{Shaded}

\begin{verbatim}
## 1.570796 with absolute error < 1.7e-14
\end{verbatim}

\hypertarget{put-functions-in-a-vector-list}{%
\section{3. Put functions in a vector
(list):}\label{put-functions-in-a-vector-list}}

\begin{Shaded}
\begin{Highlighting}[]
\NormalTok{funs <-}\StringTok{ }\KeywordTok{list}\NormalTok{(}
  \DataTypeTok{quarter =} \ControlFlowTok{function}\NormalTok{(x) x }\OperatorTok{/}\StringTok{ }\DecValTok{4}\NormalTok{,}
  \DataTypeTok{half =} \ControlFlowTok{function}\NormalTok{(x) x }\OperatorTok{/}\StringTok{ }\DecValTok{2}\NormalTok{,}
  \DataTypeTok{double =} \ControlFlowTok{function}\NormalTok{(x) x }\OperatorTok{*}\StringTok{ }\DecValTok{2}\NormalTok{,}
  \DataTypeTok{quadruple =} \ControlFlowTok{function}\NormalTok{(x) x }\OperatorTok{*}\StringTok{ }\DecValTok{4}
\NormalTok{)}
\end{Highlighting}
\end{Shaded}

\hypertarget{part-b-applied-practice}{%
\section{Part B Applied Practice}\label{part-b-applied-practice}}

\hypertarget{write-a-function-to-calculate-the-mean-that-removes-missing-data-before-conducting-the-calculation.-you-may-not-use-basemean-or-any-similar-function-that-conducts-a-mean-calculation.-include-a-warning-in-your-function-if-missing-data-have-been-removed-that-prints-the-total-number-of-cases-that-were-removed.}{%
\subsection{1. Write a function to calculate the mean that removes
missing data before conducting the calculation. You may not use
base::mean or any similar function that conducts a mean calculation.
Include a warning in your function if missing data have been removed
that prints the total number of cases that were
removed.}\label{write-a-function-to-calculate-the-mean-that-removes-missing-data-before-conducting-the-calculation.-you-may-not-use-basemean-or-any-similar-function-that-conducts-a-mean-calculation.-include-a-warning-in-your-function-if-missing-data-have-been-removed-that-prints-the-total-number-of-cases-that-were-removed.}}

\begin{Shaded}
\begin{Highlighting}[]
\NormalTok{mean1 <-}\StringTok{ }\ControlFlowTok{function}\NormalTok{(x) \{}
  \ControlFlowTok{if}\NormalTok{ (}\KeywordTok{sum}\NormalTok{(}\KeywordTok{is.na}\NormalTok{ (x) }\OperatorTok{>}\DecValTok{0}\NormalTok{ ))\{}
\NormalTok{    x_lngths <-}\KeywordTok{paste0}\NormalTok{(}\StringTok{"Number of missing case : "}\NormalTok{, }\KeywordTok{sum}\NormalTok{(}\KeywordTok{is.na}\NormalTok{ (x)))}
    \KeywordTok{warning}\NormalTok{(}\StringTok{"WARNING: Missing data have been removed, "}\NormalTok{, x_lngths)}
\NormalTok{  na <-}\StringTok{ }\KeywordTok{na.omit}\NormalTok{(x)}
  \KeywordTok{return}\NormalTok{(}
    \KeywordTok{sum}\NormalTok{(x}\OperatorTok{/}\KeywordTok{length}\NormalTok{(na), }\DataTypeTok{na.rm =}\NormalTok{ T))\}}
\ControlFlowTok{else}\NormalTok{\{(}\KeywordTok{sum}\NormalTok{(x}\OperatorTok{/}\KeywordTok{length}\NormalTok{(x)))\}}
\NormalTok{\}}
\end{Highlighting}
\end{Shaded}

\hypertarget{test-your-function-to-make-sure-it-a-provides-the-expected-results-and-b-gives-identical-output-to-basemean-when-na.rm-true.-make-sure-that-you-test-your-data-against-a-vector-that-has-missing-data.}{%
\section{2. Test your function to make sure it (a) provides the expected
results, and (b) gives identical output to base::mean when na.rm = TRUE.
Make sure that you test your data against a vector that has missing
data.}\label{test-your-function-to-make-sure-it-a-provides-the-expected-results-and-b-gives-identical-output-to-basemean-when-na.rm-true.-make-sure-that-you-test-your-data-against-a-vector-that-has-missing-data.}}

\begin{Shaded}
\begin{Highlighting}[]
\NormalTok{a <-}\StringTok{ }\KeywordTok{c}\NormalTok{(}\DecValTok{1}\NormalTok{, }\OtherTok{NA}\NormalTok{, }\OtherTok{NA}\NormalTok{, }\DecValTok{3}\NormalTok{, }\DecValTok{3}\NormalTok{, }\DecValTok{9}\NormalTok{, }\OtherTok{NA}\NormalTok{)}
\NormalTok{b <-}\StringTok{ }\KeywordTok{c}\NormalTok{(}\OtherTok{NA}\NormalTok{, }\DecValTok{3}\NormalTok{, }\OtherTok{NA}\NormalTok{, }\DecValTok{4}\NormalTok{, }\OtherTok{NA}\NormalTok{, }\OtherTok{NA}\NormalTok{, }\OtherTok{NA}\NormalTok{)}
\NormalTok{c <-}\StringTok{ }\KeywordTok{c}\NormalTok{(}\DecValTok{6}\NormalTok{,}\DecValTok{2}\NormalTok{,}\DecValTok{4}\NormalTok{,}\DecValTok{5}\NormalTok{)}

\KeywordTok{mean}\NormalTok{(a, }\DataTypeTok{na.rm =}\NormalTok{ T)}
\end{Highlighting}
\end{Shaded}

\begin{verbatim}
## [1] 4
\end{verbatim}

\begin{Shaded}
\begin{Highlighting}[]
\KeywordTok{mean1}\NormalTok{(a)}
\end{Highlighting}
\end{Shaded}

\begin{verbatim}
## Warning in mean1(a): WARNING: Missing data have been removed, Number of missing
## case : 3
\end{verbatim}

\begin{verbatim}
## [1] 4
\end{verbatim}

\begin{Shaded}
\begin{Highlighting}[]
\KeywordTok{mean}\NormalTok{(b, }\DataTypeTok{na.rm =}\NormalTok{ T)}
\end{Highlighting}
\end{Shaded}

\begin{verbatim}
## [1] 3.5
\end{verbatim}

\begin{Shaded}
\begin{Highlighting}[]
\KeywordTok{mean1}\NormalTok{(b)}
\end{Highlighting}
\end{Shaded}

\begin{verbatim}
## Warning in mean1(b): WARNING: Missing data have been removed, Number of missing
## case : 5
\end{verbatim}

\begin{verbatim}
## [1] 3.5
\end{verbatim}

\begin{Shaded}
\begin{Highlighting}[]
\KeywordTok{mean}\NormalTok{(c, }\DataTypeTok{na.rm =}\NormalTok{ T)}
\end{Highlighting}
\end{Shaded}

\begin{verbatim}
## [1] 4.25
\end{verbatim}

\begin{Shaded}
\begin{Highlighting}[]
\KeywordTok{mean1}\NormalTok{(c)}
\end{Highlighting}
\end{Shaded}

\begin{verbatim}
## [1] 4.25
\end{verbatim}

\hypertarget{turn-the-following-three-lines-of-code-into-three-different-functions.-make-sure-to-give-them-meaningful-names.-test-the-functions-to-make-sure-they-provide-the-expected-output.}{%
\subsection{3. Turn the following three lines of code into three
different functions. Make sure to give them meaningful names. Test the
functions to make sure they provide the expected
output.}\label{turn-the-following-three-lines-of-code-into-three-different-functions.-make-sure-to-give-them-meaningful-names.-test-the-functions-to-make-sure-they-provide-the-expected-output.}}

\begin{Shaded}
\begin{Highlighting}[]
\CommentTok{# Count number of missing elements in a vector}
\NormalTok{Vector_NA <-}\StringTok{ }\ControlFlowTok{function}\NormalTok{ (x) \{}
  \KeywordTok{sum}\NormalTok{(}\KeywordTok{is.na}\NormalTok{(x))\}}
\KeywordTok{Vector_NA}\NormalTok{(airquality}\OperatorTok{$}\NormalTok{Ozone)}
\end{Highlighting}
\end{Shaded}

\begin{verbatim}
## [1] 37
\end{verbatim}

\begin{Shaded}
\begin{Highlighting}[]
\CommentTok{# Proportional representation of each level (unique element) in a vector}
\NormalTok{unique_element <-}\StringTok{ }\ControlFlowTok{function}\NormalTok{(x)\{}
\NormalTok{  purrr}\OperatorTok{::}\KeywordTok{map_dbl}\NormalTok{(}\KeywordTok{split}\NormalTok{(x, x), length) }\OperatorTok{/}\StringTok{ }\KeywordTok{length}\NormalTok{(x)}
\NormalTok{\}}
\KeywordTok{unique_element}\NormalTok{(mtcars}\OperatorTok{$}\NormalTok{cyl)}
\end{Highlighting}
\end{Shaded}

\begin{verbatim}
##       4       6       8 
## 0.34375 0.21875 0.43750
\end{verbatim}

\begin{Shaded}
\begin{Highlighting}[]
\CommentTok{# normalize or z-score a vector so the mean is zero and sd is one}
\NormalTok{ZScore <-}\StringTok{ }\ControlFlowTok{function}\NormalTok{(x) \{}
\NormalTok{  (x }\OperatorTok{-}\StringTok{ }\KeywordTok{mean}\NormalTok{(x, }\DataTypeTok{na.rm =} \OtherTok{TRUE}\NormalTok{)) }\OperatorTok{/}\StringTok{ }
\StringTok{  }\KeywordTok{sd}\NormalTok{(x, }\DataTypeTok{na.rm =} \OtherTok{TRUE}\NormalTok{)}
\NormalTok{\}}

\KeywordTok{ZScore}\NormalTok{(mtcars}\OperatorTok{$}\NormalTok{mpg)}
\end{Highlighting}
\end{Shaded}

\begin{verbatim}
##  [1]  0.15088482  0.15088482  0.44954345  0.21725341 -0.23073453 -0.33028740
##  [7] -0.96078893  0.71501778  0.44954345 -0.14777380 -0.38006384 -0.61235388
## [13] -0.46302456 -0.81145962 -1.60788262 -1.60788262 -0.89442035  2.04238943
## [19]  1.71054652  2.29127162  0.23384555 -0.76168319 -0.81145962 -1.12671039
## [25] -0.14777380  1.19619000  0.98049211  1.71054652 -0.71190675 -0.06481307
## [31] -0.84464392  0.21725341
\end{verbatim}

\hypertarget{write-a-function-that-takes-a-data-frame-as-its-input-and-returns-a-data-frame-with-only-the-numeric-columns.}{%
\section{4. Write a function that takes a data frame as its input and
returns a data frame with only the numeric
columns.}\label{write-a-function-that-takes-a-data-frame-as-its-input-and-returns-a-data-frame-with-only-the-numeric-columns.}}

\begin{Shaded}
\begin{Highlighting}[]
\NormalTok{Numeric_df <-}\StringTok{ }\ControlFlowTok{function}\NormalTok{ (x)\{}
\NormalTok{nums <-}\StringTok{ }\KeywordTok{unlist}\NormalTok{(}\KeywordTok{lapply}\NormalTok{(x, is.numeric))  }
\KeywordTok{return}\NormalTok{(x[, nums])\}}

\CommentTok{# Example}
\NormalTok{d <-}\StringTok{ }\KeywordTok{data.frame}\NormalTok{(}\DataTypeTok{x1 =} \DecValTok{1}\OperatorTok{:}\DecValTok{5}\NormalTok{,                         }\CommentTok{# Create example data frame}
                   \DataTypeTok{x2 =}\NormalTok{ LETTERS[}\DecValTok{1}\OperatorTok{:}\DecValTok{5}\NormalTok{],}
                   \DataTypeTok{x3 =} \DecValTok{2}\NormalTok{,}
                   \DataTypeTok{x4 =} \KeywordTok{factor}\NormalTok{(}\KeywordTok{c}\NormalTok{(}\DecValTok{1}\NormalTok{, }\DecValTok{3}\NormalTok{, }\DecValTok{2}\NormalTok{, }\DecValTok{2}\NormalTok{, }\DecValTok{1}\NormalTok{)),}
                   \DataTypeTok{stringsAsFactors =} \OtherTok{FALSE}\NormalTok{)}

\KeywordTok{Numeric_df}\NormalTok{(d)}
\end{Highlighting}
\end{Shaded}

\begin{verbatim}
##   x1 x3
## 1  1  2
## 2  2  2
## 3  3  2
## 4  4  2
## 5  5  2
\end{verbatim}

\hypertarget{write-a-function-that-uses-the-function-you-wrote-in-question-5-and-returns-a-data-frame-with-the-mean-and-standard-deviation-of-each-numeric-column.}{%
\section{5. Write a function that uses the function you wrote in
Question 5, and returns a data frame with the mean and standard
deviation of each numeric
column.}\label{write-a-function-that-uses-the-function-you-wrote-in-question-5-and-returns-a-data-frame-with-the-mean-and-standard-deviation-of-each-numeric-column.}}

\begin{Shaded}
\begin{Highlighting}[]
\KeywordTok{library}\NormalTok{(dplyr)}
\end{Highlighting}
\end{Shaded}

\begin{verbatim}
## Warning: package 'dplyr' was built under R version 4.0.3
\end{verbatim}

\begin{verbatim}
## Registered S3 methods overwritten by 'tibble':
##   method     from  
##   format.tbl pillar
##   print.tbl  pillar
\end{verbatim}

\begin{verbatim}
## 
## Attaching package: 'dplyr'
\end{verbatim}

\begin{verbatim}
## The following objects are masked from 'package:stats':
## 
##     filter, lag
\end{verbatim}

\begin{verbatim}
## The following objects are masked from 'package:base':
## 
##     intersect, setdiff, setequal, union
\end{verbatim}

\begin{Shaded}
\begin{Highlighting}[]
\KeywordTok{library}\NormalTok{(tidyverse)}
\end{Highlighting}
\end{Shaded}

\begin{verbatim}
## Warning: package 'tidyverse' was built under R version 4.0.3
\end{verbatim}

\begin{verbatim}
## -- Attaching packages --------------------------------------- tidyverse 1.3.0 --
\end{verbatim}

\begin{verbatim}
## v ggplot2 3.3.3     v purrr   0.3.4
## v tibble  3.0.3     v stringr 1.4.0
## v tidyr   1.1.2     v forcats 0.5.0
## v readr   1.4.0
\end{verbatim}

\begin{verbatim}
## Warning: package 'ggplot2' was built under R version 4.0.5
\end{verbatim}

\begin{verbatim}
## Warning: package 'tidyr' was built under R version 4.0.3
\end{verbatim}

\begin{verbatim}
## Warning: package 'readr' was built under R version 4.0.3
\end{verbatim}

\begin{verbatim}
## Warning: package 'purrr' was built under R version 4.0.3
\end{verbatim}

\begin{verbatim}
## Warning: package 'stringr' was built under R version 4.0.3
\end{verbatim}

\begin{verbatim}
## Warning: package 'forcats' was built under R version 4.0.3
\end{verbatim}

\begin{verbatim}
## -- Conflicts ------------------------------------------ tidyverse_conflicts() --
## x dplyr::filter() masks stats::filter()
## x dplyr::lag()    masks stats::lag()
\end{verbatim}

\begin{Shaded}
\begin{Highlighting}[]
\NormalTok{summarize_cols <-}\StringTok{ }\ControlFlowTok{function}\NormalTok{(data, ...) \{}
    \KeywordTok{Numeric_df}\NormalTok{(data) }\OperatorTok\StringTok{ }
\StringTok{        }\KeywordTok{pivot_longer}\NormalTok{(}\KeywordTok{everything}\NormalTok{(), }
                     \DataTypeTok{names_to =} \StringTok{"var"}\NormalTok{, }
                     \DataTypeTok{values_to =} \StringTok{"val"}\NormalTok{) }\OperatorTok
\StringTok{        }\KeywordTok{group_by}\NormalTok{(var) }\OperatorTok
\StringTok{        }\KeywordTok{summarize}\NormalTok{(}\DataTypeTok{mean =} \KeywordTok{mean}\NormalTok{(val, }\DataTypeTok{na.rm =} \OtherTok{TRUE}\NormalTok{),}
                  \DataTypeTok{sd =} \KeywordTok{sd}\NormalTok{(val, }\DataTypeTok{na.rm =} \OtherTok{TRUE}\NormalTok{))}
\NormalTok{\}}

\KeywordTok{summarize_cols}\NormalTok{(d)}
\end{Highlighting}
\end{Shaded}

\begin{verbatim}
## # A tibble: 2 x 3
##   var    mean    sd
## * <chr> <dbl> <dbl>
## 1 x1        3  1.58
## 2 x3        2  0
\end{verbatim}

\hypertarget{if-columns-are-omitted-because-they-are-non-numeric-print-a-message-stating-the-specific-columns-that-were-not-included.}{%
\section{If columns are omitted (because they are non-numeric), print a
message stating the specific columns that were not
included.}\label{if-columns-are-omitted-because-they-are-non-numeric-print-a-message-stating-the-specific-columns-that-were-not-included.}}

\begin{Shaded}
\begin{Highlighting}[]
\NormalTok{summarize_cols_}\DecValTok{1}\NormalTok{ <-}\StringTok{ }
\StringTok{  }\ControlFlowTok{function}\NormalTok{(x)\{}
\NormalTok{    all_numeric <-}\StringTok{ }
\StringTok{      }\KeywordTok{unlist}\NormalTok{(}\KeywordTok{lapply}\NormalTok{(x, is.numeric))}
  
    \ControlFlowTok{if}\NormalTok{ (}\KeywordTok{sum}\NormalTok{(all_numeric }\OperatorTok{==}\StringTok{ }\OtherTok{FALSE}\NormalTok{) }\OperatorTok{>}\StringTok{ }\DecValTok{0}\NormalTok{)\{}
      \KeywordTok{message}\NormalTok{(}\KeywordTok{sum}\NormalTok{(all_numeric }\OperatorTok{==}\StringTok{ }\OtherTok{FALSE}\NormalTok{), }
              \StringTok{' non-numeric row(s) removed: '}\NormalTok{, }
              \KeywordTok{names}\NormalTok{(x[, }\OperatorTok{!}\NormalTok{all_numeric])}
\NormalTok{              )}
      \KeywordTok{return}\NormalTok{(}\KeywordTok{Numeric_df}\NormalTok{(x))}
\NormalTok{  \} }\ControlFlowTok{else}\NormalTok{ \{}
    \KeywordTok{Numeric_df}\NormalTok{(x)}
\NormalTok{    \}}
\NormalTok{  \}}
\end{Highlighting}
\end{Shaded}

\hypertarget{example}{%
\section{Example}\label{example}}

\begin{Shaded}
\begin{Highlighting}[]
\KeywordTok{summarize_cols_1}\NormalTok{(d)}
\end{Highlighting}
\end{Shaded}

\begin{verbatim}
## 2 non-numeric row(s) removed: x2x4
\end{verbatim}

\begin{verbatim}
##   x1 x3
## 1  1  2
## 2  2  2
## 3  3  2
## 4  4  2
## 5  5  2
\end{verbatim}

\end{document}
